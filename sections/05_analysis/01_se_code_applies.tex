%!TEX root=../../main.tex

%=========================%
% HOW THE SE CODE APPLIES %
%=========================%
\subsection{How the SE Code Applies}

\par There are many ethical codes with which we can judge Facebook's actions. It is important that we first understand why we are viewing Facebook's actions through the lens of the Software Engineering Code of Ethics and not, say, the Ten Commandments.

%---------------%
% FACEBOOK, INC %
%---------------%
% \subsubsection{Facebook, Inc.}

% \par Though we often associate it with its founder and CEO Mark Zuckerberg, Facebook is much more than that. Facebook is a corporation that includes over 18 thousand employees\footnote{As of March 31, 2017.}. \cite{fb_newsroom} 

% \par As a corporation, Facebook is legally authorized to act as a single person. \cite{mw_merriam_webster} But it is difficult to determine a single individual to hold accountable in this situation, especially as Facebook publishes multiple products unrelated to the false news controversy, such as Instagram and Messenger. \cite{fb_newsroom}

% \par The Facebook product most pertinent to this situation is its eponymous social networking service Facebook.com, the platform on which false news stories has been most noticeable. \cite{tc_responsibility}

%--------------%
% FACEBOOK.COM %
%--------------%
\subsubsection{Facebook.com}

\par One of the reasons Facebook is such a prime target for publishers of false news is the wide reach its social network provides. \cite{telegraph_fn} Facebook has so many users because it is so easily available and accessed\footnote{Grab the nearest electronic device, open a web browser, type in ``\url{www.facebook.com}'' and see for yourself.}. This is because Facebook is a website.

\par According to Merriam-Webster, a \emph{website} as ``a group of World Wide Web pages... made available online by an individual, company, educational institution, government, or organization''. \cite{merriam_webster} The ``www'' in \url{www.facebook.com} stands for ``World Wide Web'', the medium through which Facebook, Inc. makes its eponymous social networking service available.

%--------------%
% THE INTERNET %
%--------------%
\subsubsection{The World Wide Web}

% Anyone with a computer connected to the Internet can communicate with any other connected computer.\footnote{With certain limitations. Internet standards dictate that connected computers should restrict access to certain entities to maintain privacy. \cite{cpe_464}} \cite{internet}

\par The \emph{Internet}, or \emph{World Wide Web}, is a global network that connects computers. \cite{cpe_464} Though the physical manifestation of the Internet does involve networking devices and cables, we use the term ``Internet'' to refer to the communication methods and protocols used to transmit data packets along these cables. \cite{cpe_464}

%----------%
% SOFTWARE %
%----------%
%\subsubsection{The Intangible}

\par Merriam-Webster defines \emph{software} as ``something used or associated with and usually contrasted with hardware; the entire set of programs, procedures, and related documentation associated with a... computer system''. \cite{merriam_webster} The Internet, then, is the software used with, associated with, and contrasted with the hardware of cables and routers that connect our computers. 

\par Facebook.com, as a website, is built upon the Internet. \citeneeded 


%------------%
% DEVELOPERS %
%------------%
\subsubsection{As Developers}

\par In its preamble, the Software Engineering Code of Ethics defines \emph{software engineers} as ``those who contribute by direct participation or by teaching, to the analysis, specification, design, development, certification, maintenance, and testing of software systems'' 

\par \textbf{BLAH BLAH FACEBOOK} are then subject to the Software Engineering Code of Ethics and ``shall adhere to the code''. \cite{se_code}

%--------------%
% POLICYMAKERS %
%--------------%
\subsubsection{As Policymakers}

\par The Software Engineering Code of Ethics further states that its 8 principles are ``related to the behavior of and decisions made by professional software engineers, including practitioners, educators, managers, supervisors and policy makers''.
 
\par The Oxford Dictionary defines a policy as ``a course or principle of action adopted or proposed by a government, party, business, or individual''. \cite{oxford} In addition to things like ``updat[ing] its policy to rule out fake news sites from using Facebook Audience Network'', \cite{tc_ban} Facebook is establishing policies whenever it decides to ``disrupt financial incentives for spammers'', \cite{fb_hoaxes_2016}

\par Therefore, Facebook plays a role 

What Facebook is doing in addressing false news is establishing principles of action --- for exampl