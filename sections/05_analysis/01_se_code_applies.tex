%!TEX root = ../../main.tex


%=========================%
% HOW THE SE CODE APPLIES %
%=========================%
\subsection{How the SE Code Applies}

\par Discussion of ethics and ethicality can be muddy. Philosophers have argued over virtue and morality since Classical Greece.\citeneeded

\par Society has generated many ethical codes and legal systems. It is important that we first understand why we are judging Facebook's actions through the lens of the Software Engineering Code of Ethics and not, say, the Ten Commandments.

%---------------%
% FACEBOOK, INC %
%---------------%
\subsubsection{Facebook, Inc.}

\par Though we often associate it with its founder and CEO Mark Zuckerberg, Facebook is much more than that. Facebook is a corporation\footnote{Hence its ``Incorporated'' company suffix.} that includes over 18 thousand employees\footnote{As of March 31, 2017.}. \cite{fb_about} 

\par As a corporation, Facebook is legally authorized to act as a single person. \cite{dictionary} But it is difficult to determine a single individual to hold accountable in this situation, especially as Facebook publishes multiple products unrelated to the false news controversy, such as Instagram and Messenger. \cite{fb_about}

\par The Facebook product most pertinent to this situation is its eponymous social networking service Facebook.com, the platform on which false news stories has been most noticeable. \cite{tc_facebook_responsibility}

%--------------%
% FACEBOOK.COM %
%--------------%
\subsubsection{Facebook.com}

\par One of the reasons Facebook is such a prime target for publishers of false news is the wide reach its social network provides. \cite{telegraph_fake_news} Facebook has so many users because it is so easily available and accessed\footnote{Grab the nearest electronic device, open a web browser, type in ``\url{www.facebook.com}'' and see for yourself.}. This is because Facebook is a website.

\par A \emph{website} is a group of World Wide Web pages made available by an individual, company, educational institution, government, or organization. \cite{dictionary} The ``www'' in \url{www.facebook.com} stands for ``World Wide Web'', the medium with which Facebook, Inc. publishes its eponymous social media platform.

%--------------%
% THE INTERNET %
%--------------%
\subsubsection{The World Wide Web}

% Anyone with a computer connected to the Internet can communicate with any other connected computer.\footnote{With certain limitations. Internet standards dictate that connected computers should restrict access to certain entities to maintain privacy. \cite{cpe464}} \cite{internet}

\par The \emph{Internet}, or \emph{World Wide Web}, is a global network that connects computers. \cite{internet} Though the Internet does involve networking devices and cables, we use the term ``Internet'' to refers to the communication methods and protocols used to transmit data packets\footnote{The idea of packet switching is central to the concept of Internet and differentiates it from other modes of communication like telephones. \cite{cpe464}} along these cables. \cite{cpe464}

\par It is built on the vision of J.C.R. Licklider\footnote{M.I.T. researcher, first head of DARPA's computer research program, and in general one of the most important figures in modern computing. \cite{internet}}, who envisioned a ``globally interconnected set of computers through which everyone could quickly access data and programs from any site''. As a mere description of the aforementioned communication methods and protocols, the Internet is nothing more than a concept.

%----------%
% SOFTWARE %
%----------%
\subsubsection{The Intangible}

\par According to Merriam-Webster, \emph{software} is ``something used or associated with and usually contrasted with hardware; the entire set of programs, procedures, and related documentation associated with a... computer system''. \cite{dictionary} The Internet, then, is the software contrasted with the hardware of cables and routers that connect our computers.

\par Facebook began as an idea. Mark Zuckerberg wanted to \textbf{do something}. \citeneeded To turn this concept into a reality he used the Internet as a foundation for his website, Facebook.com. Facebook.com is nothing more than a set of programs, procedures, and related documentation that, when compiled, allow users to chat with friends and share content over the interwebs. What started as an idea became implemented as one of the most recognizable pieces of software \textbf{something}. \citeneeded


%------------%
% DEVELOPERS %
%------------%
\subsubsection{The Creators}

% 
\par 

\par It is further constructed of several smaller products; Facebook's News Feed, Facebook's Profiles, and Facebook's Pages are all separate software projects developed by Facebook, Inc. \cite{fb_about}

\par 

\par \textbf{BLAH FUCKING BLAH FACEBOOK} are then subject to the Software Engineering Code of Ethics and ``shall adhere to the code.'' \cite{se_code}

\par That aside, Facebook is one of the global leaders in the tech industry. Its employees --- whether or not ``software engineer'' appears on their business cards --- must understand thay they set standards for the profession of software engineering as a whole.
