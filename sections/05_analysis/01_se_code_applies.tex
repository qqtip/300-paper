%!TEX root = ../../main.tex


%-------------------------%
% HOW THE SE CODE APPLIES %
%-------------------------%
\subsection{How the SE Code Applies}

\par In its preamble, the ACM Software Engineering Code of Ethics defines software engineers as ``those who contribute by direct participation or by teaching, to the... design, development, certification, maintenance and testing of software systems'' and that it applies to ``the behavior of and decisions made by... managers, [and] supervisors'' of the Software Engineering profession. \cite{se_code}

%
% show direct participation
% define software
% define design, maintenance, involvement and prove it
% define "software system"
%

\subsubsubsection{The Software}

\par Facebook's eponymous social media n

\subsubsubsection{Facebook}

\par On the other hand, First Draft News' official announcement of CrossCheck states that Facebook's support comes in the form of providing dedicated tools and an interface with users through their social network. \cite{crosscheck_launch} Coincidentally, at the same time as First Drafts' unveiling of CrossCheck, Facebook launched its own measures to combat fake news in allowing its users to report articles and dispute their validity. \cite{tc_google_fb_partner} Facebook, as the analyzers, specifiers, designers, developers, certifiers, maintainers, and testers of their eponymous social media platform and the analytics software behind CrossCheck, are then subject to the Software Engineering Code of Ethics and ``shall adhere to the code.'' \cite{se_code}

\par Regardless of the nature of either tech giant's involvement in the platform, those at Google and Facebook --- whether or not ``software engineer'' appears on their business cards --- must understand that, as two of the largest names on the internet , any policies they espouse will become standards for the profession of software engineering as a whole.

\subsubsubsection{Definitions}

\begin{enumerate}

\item{\textbf{Product}: The \emph{product} in this case will refer to social media platform Facebook, which is a piece of software developed by Mark Zuckerberg and his employees.
}

\item{\textbf{Software Engineer}: The \emph{software engineers} in this case are those who develop, test, and maintain Facebook: Mark Zuckerberg and his employees.
}

\item{\textbf{Users}: The \emph{users} in this case are people who have registered for Facebook and who browse the platform at least weekly.
}

\item{\textbf{Fake News:} One of the difficulties in identifying ``fake news'' is semantic: President Donald Trump's liberal usage the term during the 2016 Presidential Election has obscured its meaning. \cite{npr_fake_news} To remedy this, we will define the following:
\begin{enumerate}
    \item{\emph{False News}: A story that is not based in fact that may reports events that did not happen. A false news story is published with the intent to deceive. \cite{ted_becka}}
    \item{\emph{Satire}: A published story that satirizes news media. \cite{ted_becka}}
    \item{\emph{News with Opinion}: An editorial piece that covers news and introduces the bias of the writer. \cite{ted_becka}}
\end{enumerate}
}

\end{enumerate}
