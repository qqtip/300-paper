%!TEX root=../../main.tex

%=========================%
% HOW THE SE CODE APPLIES %
%=========================%
\subsection{How the SE Code Applies}

\par There are many ethical codes with which we can judge Facebook's actions. It is important that we first understand why we are viewing Facebook's actions through the lens of the Software Engineering Code of Ethics and not, say, the Ten Commandments.

%---------------%
% FACEBOOK, INC %
%---------------%
% \subsubsection{Facebook, Inc.}

% \par Though we often associate it with its founder and CEO Mark Zuckerberg, Facebook is much more than that. Facebook is a corporation that includes over 18 thousand employees\footnote{As of March 31, 2017.}. \cite{fb_newsroom} 

% \par As a corporation, Facebook is legally authorized to act as a single person. \cite{mw_merriam_webster} But it is difficult to determine a single individual to hold accountable in this situation, especially as Facebook publishes multiple products unrelated to the false news controversy, such as Instagram and Messenger. \cite{fb_newsroom}

% \par The Facebook product most pertinent to this situation is its eponymous social networking service Facebook.com, the platform on which false news stories has been most noticeable. \cite{tc_responsibility}

%--------------%
% FACEBOOK.COM %
%--------------%
\subsubsection{Facebook.com}

\par One of the reasons Facebook is such a prime target for publishers of false news is the wide reach its social network provides. \cite{telegraph_fn} Facebook has so many users because it is so easily available and accessed\footnote{Grab the nearest electronic device, open a web browser, type in ``\url{www.facebook.com}'' and see for yourself.}. This is because Facebook is a website.

\par According to Merriam-Webster, a \emph{website} as ``a group of World Wide Web pages... made available online by an individual, company, educational institution, government, or organization''. \cite{merriam_webster} The ``www'' in \url{www.facebook.com} stands for ``World Wide Web'', the medium through which Facebook, Inc. makes its eponymous social networking service available.

\par In addition, Facebook.com is not a static, unchanging set of pages; it is built upon a dynamic set of software libraries and frameworks, some of which Facebook develops itself. \cite{fb_code} Facebook's News Feed is not a singular application running on a singular computer. The social network is just that: a network. It is built upon its web of users.  Indeed, this interconnected set of working parts in the Facebook network make it a system. \cite{oxford}

%------------%
% DEVELOPERS %
%------------%
\subsubsection{As Developers}

\par In its preamble, the Software Engineering Code of Ethics defines \emph{software engineers} as ``those who contribute by direct participation or by teaching, to the analysis, specification, design, development, certification, maintenance, and testing of software systems'' 

\par Because Facebook.com and the Facebook News Feed are software systems designed and developed by the engineers at Facebook, Inc., those involved are then subject to the Software Engineering Code of Ethics and ``shall adhere to the code''. \cite{se_code}

%-------------%
% MAINTAINERS %
%-------------%
\subsubsection{As Maintainers}

\par Facebook's software engineers designed and developed the Facebook News Feed, which makes them historically subject to the Software Engineering Code of Ethics. At this time, they are focusing on preserving the software by providing upkeep in response to the false news epidemic. This is how the Merriam-Webster dictionary describes maintenance. \cite{merriam_webster} Facebook is then subject to the Code of Ethics under its current role as software maintainers.

%--------------%
% POLICYMAKERS %
%--------------%
\subsubsection{As Policymakers}

\par The Software Engineering Code of Ethics further states that its 8 principles are ``related to the behavior of and decisions made by professional software engineers, including practitioners, educators, managers, supervisors and policy makers''. \cite{se_code}
 
\par The Oxford Dictionary defines a policy as ``a course or principle of action adopted or proposed by a government, party, business, or individual''. \cite{oxford} In addition to things like ``updat[ing] its policy to rule out fake news sites from using Facebook Audience Network'', \cite{tc_ban} Facebook is establishing policies whenever it decides to ``disrupt financial incentives for spammers''. \cite{fb_hoaxes_2016} Facebook is establishing principles of actions in response to false news and is thus establishing policy.

\par Facebook is one of the largest tech companies in the world and is addressing a new problem --- how they react will establish precedents for responding to false news in the future. Facebook has higher responsibility.