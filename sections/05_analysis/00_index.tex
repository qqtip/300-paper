%!TEX root = ../../main.tex

\newcommand{\tenetbox}[1]{
    \begin{framed}
    \emph{#1}
    \end{framed}
    \vspace{2mm}
}

\newcommand{\tenetsection}[1]{
    \subsubsection{#1}
}

\newcommand{\dsr}[1]{
    \subsubsubsection{Domain Specific Rule}
    \begin{framed}
    \emph{#1}
    \end{framed}
}

%----------%
% ANALYSIS %
%----------%
\section{Analysis}
%-----------------%
% SE CODE APPLIES %
%-----------------%
%!TEX root = ../../main.tex


%-------------------------%
% HOW THE SE CODE APPLIES %
%-------------------------%
\subsection{How the SE Code Applies}

\par In its preamble, the ACM Software Engineering Code of Ethics defines software engineers as ``those who contribute by direct participation or by teaching, to the analysis, specification, design, development, certification, maintenance and testing of software system'' and that it applies to ``the behavior of and decisions made by practicioners... managers, supervisors, and policymakers'' of the Software Engineering profession. \cite{se_code}

\par The entity behind CrossCheck, the collaborative news verification project in question, is First Draft News, a nonprofit coalition that counts Google --- through Google News Lab --- as one of its founding partners. Google News Lab provided some of the financial backing behind \url{firstdraftnews.com}, supporting its web presence. \cite{firstdraftnews_about} While Google has recently faced controversy in removing fake news from its ad network, \cite{tc_google_took_down_ads} it seems that its involvement in CrossCheck is largely financial.

\par On the other hand, First Draft News' official announcement of CrossCheck states that Facebook's support comes in the form of providing dedicated tools and an interface with users through their social network. \cite{crosscheck_launch} Coincidentally, at the same time as First Drafts' unveiling of CrossCheck, Facebook launched its own measures to combat fake news in allowing its users to report articles and dispute their validity. \cite{tc_google_fb_partner} Facebook, as the analyzers, specifiers, designers, developers, certifiers, maintainers, and testers of their eponymous social media platform and the analytics software behind CrossCheck, are then subject to the Software Engineering Code of Ethics and ``shall adhere to the code.'' \cite{se_code}

\par Regardless of the nature of either tech giant's involvement in the platform, those at Google and Facebook --- whether or not ``software engineer'' appears on their business cards --- must understand that, as two of the largest names on the internet , any policies they espouse will become standards for the profession of software engineering as a whole.

\subsubsubsection{Definitions}

\begin{enumerate}

\item{\textbf{Product}: The \emph{product} in this case will refer to social media platform Facebook, which is a piece of software developed by Mark Zuckerberg and his employees.
}

\item{\textbf{Software Engineer}: The \emph{software engineers} in this case are those who develop, test, and maintain Facebook: Mark Zuckerberg and his employees.
}

\item{\textbf{Users}: The \emph{users} in this case are people who have registered for Facebook and who browse the platform at least weekly.
}

\item{\textbf{Fake News:} One of the difficulties in identifying ``fake news'' is semantic: President Donald Trump's liberal usage the term during the 2016 Presidential Election has obscured its meaning. \cite{npr_fake_news} To remedy this, we will define the following:
\begin{enumerate}
    \item{\emph{False News}: A story that is not based in fact that may reports events that did not happen. A false news story is published with the intent to deceive. \cite{ted_becka}}
    \item{\emph{Satire}: A published story that satirizes news media. \cite{ted_becka}}
    \item{\emph{News with Opinion}: An editorial piece that covers news and introduces the bias of the writer. \cite{ted_becka}}
\end{enumerate}
}

\end{enumerate}

%---------------------%
% PRINCIPLE 1: PUBLIC %
%---------------------%
%!TEX root = ../../main.tex


%------------%
% 1. PUBLIC  %
%------------%
\subsection{Principle 1: Public}

%------%
% 1.02 %
%------%
\subsubsection{Tenet 1.02}
\emph{\ul{Moderate} the \ul{interests} of the \ul{software engineer}, the \ul{employer}, the \ul{client} and the \ul{users} with the \ul{public good}. \cite{se_code}}

\tenetmargin

\par In the false news debate, Mark Zuckerberg and Facebook's other employees are the \ul{software engineers} involved, as they design, develop, maintain, and test the social media platform on which false news circulates. As Facebook develops its own product, Mark Zuckerberg and these same software engineers also comprise the \ul{employer}. 

\par Facebook's users, which include both readers and purveyors of false news, comprise the \ul{users} in this situation. As the end customers of the platform, these users are also Facebook's \ul{clients}.

\par While Facebook affects much of the public due to its size and reach, here the most important aspect of \ul{public good} being affected is the wide disinformation of its \ul{users}.

\par To \emph{moderate} something means to preside over and to bring balance to its extremities. \cite{dictionary} The \ul{interests} of these disparate groups are their various wants and needs. For Facebook to moderate their interests means for Facebook to preside over and balance these various wants and needs. 

\subsubsubsection{Domain Specific Rule}

Applying Tenet 1.02 to our domain suggests that Facebook should \emph{balance their needs with the needs and well-being of their users}.
% TODO

%------%
% 1.04 %
%------%
\subsubsection{Tenet 1.04}
\emph{\ul{Disclose} to \ul{appropriate persons or authorities} any \ul{actual or potential danger} to \ul{the user, the public}, or the environment, that they reasonably believe to be \ul{associated with software or related documents}. \cite{se_code}}

\tenetmargin

In this situation, the \ul{appropriate persons or authorities} are the fact-checkers behind CrossCheck, \ul{actual or potential danger} refers to misinformation, and \ul{the user and public} are Facebook users and readers of news.

\subsubsubsection{Domain Specific Rule}

Applying Tenet 1.04 to our domain suggests that
% TODO

%----------------------------------%
% PRINCIPLE 2: CLIENT AND EMPLOYER %
%----------------------------------%
%!TEX root = ../../main.tex

\subsection{Principle 2: Client and Employer}

%------%
% 2.07 %
%------%
\subsubsection{Tenet 2.07}

\tenetbox{Identify, document, and report significant \ul{issues of social concern}, of which they are aware, in \ul{software} or related documents, to the employer or the client. \cite{se_code}}

In this situation, the \ul{issues of social concern} are the widespread misinformation of the public and \ul{software} refers to the social media network platform.

\dsr{blah}
% TODO

%----------------------%
% PRINCIPLE 3: PRODUCT %
%----------------------%
%!TEX root = ../../main.tex

\subsection{Facebook and the Product}

%------%
% 3.03 %
%------%
\subsubsection{Tenet 3.03}

\tenetbox{\ul{Identify, define and address} ethical, economic, cultural, legal and environmental issues related to work projects. \cite{se_code}}

\par As software engineers, those involved are expected to understand that their work has implications in their world at large.

\dsr{Blah}
% TODO

%-------------------------%
% PRINCIPLE 7: COLLEAGUES %
%-------------------------%
%!TEX root = ../../main.tex


%----------------%
% 7. COLLEAGUES  %
%----------------%
\subsection{Principle 7: Colleagues}

%------%
% 7.04 %
%------%
\subsubsection{Tenet 7.04}
\emph{\ul{Review} the \ul{work} of \ul{others} in an \ul{objective}, \ul{candid}, and \ul{properly-documented} way. \cite{se_code}}

\tenetmargin

\par The \emph{work} here refers to articles not published by Facebook. These articles are published by \emph{others} --- namely online journalists, real or fake.

\par For Facebook to \emph{review} these articles means to determine the accuracy and value of said articles. To be \emph{objective} or \emph{candid} in their review means Facebook should analyze this work without bias. To be \emph{properly-documented} in review means Facebook should be open and transparent about their review process. \cite{dictionary}

\subsubsubsection{Domain Specific Rule}

Applying Tenet 7.04 to our domain states that Facebook should \emph{determine the accuracy of online publishers' news articles without bias and be transparent in their methods}.

%------%
% 7.08 %
%------%
\subsubsection{Tenet 7.08}
\emph{In \ul{situations outside of their own areas of competence}, 
\ul{call upon the opinions} of \ul{other professionals} who have \ul{competence in that area}. \cite{se_code}}

\tenetmargin

\par Facebook software engineers' \ul{areas of competence} revolve around their work --- software engineering. This \ul{situation} revolves around false news and so is a situation outside of Facebook's area of competence.

\par Those with \ul{competence in the area} of news would be journalists, who have dedicated their career to collecting and editing news. \cite{dictionary} Therefore, journalists are the \ul{other professionals} who are competent in this area.

\par For Facebook's software engineers to \ul{call upon the opinions} of these journalists would mean to consult them in regards to the subject. An \ul{opinion} is a belief or judgment about a particular matter, \cite{dictionary} and journalists do provide editorial statements in their line of work. However, in this case it would be more appropriate for Facebook to consult journalists for their \emph{expertise} and \emph{fact-checking experience} on the matter.

\subsubsubsection{Domain Specific Rule}

Applying Tenet 7.08 to our domain suggests that Facebook should, \emph{in situations related to news, consult the expertise of journalists who specialize in that area}. 


\subsection{The Intent of False News Publishers}

\par Once we have distinguished false news from satire and opinion pieces, it is important to discuss the intent of those who publish such stories. Unfortunately, there is no catch-all case for these publishers.

\par Barring AdBlock, it is not possible to browse the internet without being bombarded by (possibly-targeted) ads. As internet usership grew, so did the efficacy of ad space on the internet. Ads are now widespread and the ultimate goal for some websites is monetization through adspace. \cite{bbc_clickbait} So for many publishers of false news, the motivation is financial --- the intent is to receive clicks. Due to the capitalization of the internet, online writers have turned to incitory headlines and content with the goal of receiving clicks. This is one source of fake news.

\par On April 27th, Facebook published an official transparency report on usage of the social media platform for \emph{information operations}. In this case, Facebook defines \emph{information operations} as ``Actions taken by governments or organized non-state
actors to distort domestic or foreign political sentiment''. \cite{fb_info_ops} Information operations includes the dissemination of false news stories (via bots or otherwise) through social media to disinform the public. Here the motivation is political and the intent is to actively manipulate the beliefs of Facebook's users for personal gain. This amounts to lying, and Kantian ethics could not apply lying as a rule. \cite{kant} Therefore, the usage of fake news for information operations is entirely unethical.

\par Some publishers of fake news have a skewed self-perspective, viewing themselves as fiction writers or prophets. \cite{cbs_fake_news}


\subsection{Precedents}

\subsubsection{Burnett v. National Enquirer}

\par TODO: case against National Enquirer by Carol Burnett for libel. National Enquirer paid settlement out of court. Established the possibility of celebrities to sue tabloids for defamation but also established that the national enquirer is not a real newspaper and is not subject to restrictions on journalism.
\cite{burnett_v_ne}
