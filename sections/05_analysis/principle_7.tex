%!TEX root=../../main.tex

%===================%
% FACEBOOK AND BIAS %
%===================%
\subsection{Facebook's Bias}

\par We have established that Facebook is acting in the best interest of its users. Now we must explore whether their actions themselves are ethical.

\par Central to this analysis is identifying whether Facebook's actions are biased. Some 

%------%
% 7.04 %
%------%
\subsubsection{Tenet 7.04}

\tenetbox{\ul{Review} the \ul{work} of \ul{others} in an \ul{objective}, \ul{candid}, and \ul{properly-documented} way.}

\par According to the Oxford English Dictionary, to \ul{review} is to ``\emph{Assess} (something) formally with the intention of instituting change if necessary''. \cite{oxford} The \ul{work} being reviewed or \emph{assessed} is the body of \emph{articles} published by online journalists and publishers, real or fake. While these journalists share their work on Facebook's social network, they are not affiliated with Facebook. Therefore these \emph{journalists} are the \ul{others}.

\par We have that Facebook is expected to assess articles published on its social network, but the Tenet further specifies \emph{how} Facebook should act. 

\par To be \ul{objective} is to ``not [be] influenced by personal feelings or opinions in considering and representing facts'', \cite{oxford} to be \ul{candid} is to be ``truthful and straightforward'', \cite{oxford} and to \ul{document} something is to ``record (something) in written... form''. \cite{oxford}

\par So Facebook's methods should not be influenced by personal feelings or \emph{biases}. Their assessment of articles must be based on truth or \emph{accuracy} and not opinion. In addition, they must be truthful in recording their methods. For Facebook to be truthful and record their methods requires Facebook to be transparent, or \emph{open}. Thus, we can apply Tenet 7.04 of the Software Engineering Code of Ethics to our domain:

\dsr{7.04}{Facebook shall assess the accuracy of online publishers' news articles without bias and be open with their methods.}

%-------------------------%
% ASSESSMENT AND ACCURACY %
%-------------------------%
\subsubsection{Accuracy in the Post-Truth Climate}

\par A brief aside on what Facebook is assessing

\subsubsubsection{What is fake news?}

\par Talk about the types of fake news. \cite{ted_becka}

\par We have been concerned with clickbait since before the election.

\par Now we are more concerned with false news.

\subsubsubsection{Assessment Criteria}

\par What makes a news story false?

\par How does Facebook determine accuracy? \cite{tc_ban}

\subsubsubsection{Facebook's Assessment}

\par How is Facebook assessing these articles?

\par Segue into methods

%---------%
% METHODS %
%---------%
\subsubsection{Facebook's Methods}

\par Did Facebook act openly?

\subsubsubsection{Transparency}

\par What exactly would constitute transparency?

\par Criticism

\par Something about announcements \cite{fb_hoaxes_2015}

\par Something about transparency report \cite{fb_info_ops}

\par So yes, they were transparent.

\subsubsubsection{Methods}

\par What exactly \emph{were} Facebook's methods?

\par Something about announcements

\par Segue into bias

%------%
% BIAS %
%------%
\subsubsection{The Bias Inherent in the System}

\par Was Facebook biased? Let's find out.

\subsubsubsection{Early Efforts}

\par I guess this is a preface.

\subsubsubsection{In-House Curation}

\par Facebook had an in-house curation team.
\par Allegations that it skewed liberal and targetted conservative stories.
\par Facebook denies, but fired its team. \cite{tc_ai}

\subsubsubsection{A Try with AI}

\par Facebook also used AI. \cite{tc_ai}

\par So did Google.

\par Could not because reasons. \cite{ai_cannot_win}

\subsubsubsection{Independent Fact-Checkers}

\par Did Facebook attempt to be impartial? How?

\par Outside fact checkers \cite{tc_downranks}

\par Fact checkers not credible \cite{vox_sentences}

\par Whether or not they are credible, Facebook worked with them, which is responsible.

\par Segue

\subsubsubsection{Conclusion}

\par They tried their best.

%------%
% 7.08 %
%------%
\subsubsection{Tenet 7.08}

\tenetbox{In \ul{situations outside of their own areas of competence}, 
\ul{call upon the opinions} of \ul{other professionals} who have \ul{competence in that area}.}

\par Facebook software engineers' \ul{areas of competence} revolve around their work --- software engineering. This \ul{situation}, the issue of false news, is outside of Facebook's area of competence.

\par Those with \ul{competence in the area} of news would be journalists, who have dedicated their career to collecting and editing news. \cite{merriam_webster} Therefore, journalists are the \ul{other professionals} who are competent in this area.

\par For Facebook's software engineers to \ul{call upon the opinions} of these journalists would mean to consult them in regards to the subject. An \ul{opinion} is a belief or judgment about a particular matter, \cite{merriam_webster} and journalists do provide editorial statements in their line of work. However, in this case it would be more appropriate for Facebook to consult journalists for their \emph{expertise} and \emph{fact-checking experience} on the matter.

\dsr{7.08}{Facebook must, in the fake news controversy, consult the expertise of journalists who specialize in that area.}

%-
% 
%-
\subsubsection{The Situation/Controversy}

\par What is going on?

\subsubsubsection{Internet Journalism}

\par Talk about the shifting landscape of journalism.

\subsubsubsection{Facebook's Area of Competence}

\par Journalism is outside of Facebook's area of competence.

%-
%
%-
\subsubsection{Traditional Journalism}

\par Who are the experts?

\subsubsubsection{Staying Relevant}

\par Are they??

%-
%
%-
\subsubsection{Facebook's Collaborations}

\par How did Facebook work with journalists?

\subsubsubsection{Start}

\par Full Fact, First Draft, others

\subsubsubsection{A Means or an End}\label{7_intent}

\par As discussed in section \ref{2_intent}, we will need determine Facebook's motives in addressing the false news issue. It is
\cite{kant}