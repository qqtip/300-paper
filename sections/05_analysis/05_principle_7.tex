%!TEX root = ../../main.tex

\subsection{Facebook and its Colleagues}



%------%
% 7.04 %
%------%
\subsubsection{Tenet 7.04}

\tenetbox{\ul{Review} the \ul{work} of \ul{others} in an \ul{objective}, \ul{candid}, and \ul{properly-documented} way. \cite{se_code}}

\par The \emph{work} here refers to articles. These articles are published by \emph{others} --- namely online journalists, real or fake. While they are spread on Facebook's social network, Facebook is not directly involved in their production.

\par For Facebook to \emph{review} these articles means to determine the accuracy and value of said articles. To be \emph{objective} or \emph{candid} in their review means Facebook should analyze this work without bias. To be \emph{properly-documented} in review means Facebook should be open and transparent about their review process. \cite{dictionary}

\dsr{Facebook must determine the accuracy of online publishers' news articles without bias and be transparent in their methods.}

%-
%
%-
\subsubsection{Accuracy in the Post-Truth Climate}

\par How does Facebook determine accuracy?

%-
%
%-
\subsubsection{The Bias Inherent in the System}

\par Was Facebook biased?

%-
%
%-
\subsubsection{Facebook's Methods}

\par Did Facebook act openly?

%------%
% 7.08 %
%------%
\subsubsection{Tenet 7.08}

\tenetbox{In \ul{situations outside of their own areas of competence}, 
\ul{call upon the opinions} of \ul{other professionals} who have \ul{competence in that area}. \cite{se_code}}

\par Facebook software engineers' \ul{areas of competence} revolve around their work --- software engineering. This \ul{situation}, the issue of false news, is outside of Facebook's area of competence.

\par Those with \ul{competence in the area} of news would be journalists, who have dedicated their career to collecting and editing news. \cite{dictionary} Therefore, journalists are the \ul{other professionals} who are competent in this area.

\par For Facebook's software engineers to \ul{call upon the opinions} of these journalists would mean to consult them in regards to the subject. An \ul{opinion} is a belief or judgment about a particular matter, \cite{dictionary} and journalists do provide editorial statements in their line of work. However, in this case it would be more appropriate for Facebook to consult journalists for their \emph{expertise} and \emph{fact-checking experience} on the matter.

\dsr{Facebook must, in the false news controversy, consult the expertise of journalists who specialize in that area.}

%-
%
%-
\subsubsection{Internet Journalism}

\par What is going on?

%-
%
%-
\subsubsection{Traditional Journalism}

\par Who are the experts?

%-
%
%-
\subsubsection{Facebook's Collaborations}

\par How did Facebook work with journalists?
