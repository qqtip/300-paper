%!TEX root=../../main.tex

\subsection{Facebook and its Users}

\par Principle 2 of the Software Engineering Code of Ethics is concerned with the relationship between the software engineer and the client.\footnote{Principle 2 also addresses the relationship between the software engineer and employer. However, CEO of Facebook Mark Zuckerberg (who would be our employer) is an actor in this case, so we will omit him from our analysis.} \cite{se_code} Before delving into the Code, we must first establish who exactly Facebook's clients are.

\par Facebook's News Feed, the software product in question, is free for users. Any user only needs to log in to his or her account to view trending stories. And News Feed is not the only product Facebook provides to users free of charge. All of Facebook's other user applications, such as Facebook Profile, Messenger, and Groups, are free for users. So is Instagram, Facebook Inc.'s other major product. \cite{fb_newsroom}

\par With over a billion daily users consuming bandwidth and utilizing server resources, Facebook is extremely reliant on ad revenue. In fact, a whopping 97\% of its revenue in 2016 came from advertisements. \cite{fb_4q2016} One wonders where Facebook's loyalties lie.

%---------%
% MISSION %
%---------%
\subsubsection{Facebook's Mission}\label{mission}

\par Taken from its official website, Facebook's mission ``is to give people the power to share and make the world more open and connected. People use Facebook to stay connected with friends and family, to discover what’s going on in the world, and to share and express what matters to them.'' \cite{fb_newsroom}

\par According to its mission statement, Facebook works for its users. But as a business, Facebook requires its relationship with advertisers. In fact, TechCrunch's Jordan Crook claims that Facebook \emph{gains} from the appearance of false news in its ad network.  \cite{tc_responsibility}

\par In November 2016, among the responses Facebook enacted against false news was banning false news websites from its targeted advertising program, the Facebook Audience Network. \cite{tc_ban} The next fiscal quarter Facebook saw a 9\% decrease in revenue.\footnote{This follows a 29\% increase over the last two quarters.} \cite{tc_ban,fb_4q2016,fb_1q2017} 

\par The software product at hand, Facebook.com (or more specifically, the Facebook News Feed), is free for users. Although users do not need to pay to use Facebook.com, Facebook's primary goal is to serve the user. Facebook's primary client, then, is the average Facebook user.\footnote{False news websites on Facebook are also technically users, but they are a minority.}

%------%
% 2.07 %
%------%
\subsubsection{Tenet 2.07}

\tenetbox{\ul{Identify}, \ul{document}, and \ul{report} \ul{significant issues} of \ul{social concern}, of which they are aware, in \ul{software or related documents}, to the \ul{employer or the client}. \cite{se_code}}

\par At the core of this tenet is the idea of a \ul{significant issue} of \ul{social concern} and how Facebook, as a software engineer, should handle it.

\par According to the Oxford English Dictionary, an \ul{issue} is ``[a]n \emph{important topic or problem} for debate''. \cite{oxford} This makes \ul{significant} redundant for our purposes. A \ul{concern} is ``[a] matter of interest or importance to \ul{someone}'', \cite{oxford} which further highlights the \emph{importance} of the \emph{problem} at hand. In addition, it specifies an actor, a \ul{someone} to whom the \emph{problem} is important. Now we can focus on the \ul{social} aspect of the \emph{important problem}.

\par Almost two-thirds of American adults believe that false news is causing confusion about current events. \cite{bbc_propaganda} The false news epidemic reaches far beyond American borders as well, making headlines in Germany, \cite{wp_germany} France, \cite{crosscheck} and the United Kingdom. \cite{guardian_steps} So not only is the problem of false news important to American society, if affects the world. 

to \ul{identify} something is tp ``[e]stablish or indicate who or what (someone or something) is.'' \cite{oxford} To \ul{document} is to ``[r]ecord (something) in written, photographic, or other form.'' \cite{oxford} To \ul{report} is to ``[g]ive a spoken or written account of something that one has observed, heard, done, or investigated.'' \cite{oxford}

\par In this situation, the \ul{issues of social concern} are the widespread misinformation of the public and \ul{software} refers to the social media network platform.

\par Facebook is not answering to an \ul{employer} in this situation. CEO Mark Zuckerberg is deeply involved. As discussed in \ref{mission}, the \ul{client} is the average Facebook.com user. 

%-----%
% DSR %
%-----%
\dsr{Facebook must identify the issue of false news and report to its users.}




\subsubsection{The Intent of False News Publishers}

\par Once we have distinguished false news from satire and opinion pieces, it is important to discuss the intent of those who publish such stories. Unfortunately, there is no catch-all case for these publishers.

\par Barring AdBlock, it is not possible to browse the internet without being bombarded by (possibly-targeted) ads. As internet usership grew, so did the efficacy of ad space on the internet. Ads are now widespread and the ultimate goal for some websites is monetization through adspace. \cite{bbc_clickbait} So for many publishers of false news, the motivation is financial --- the intent is to receive clicks. Due to the capitalization of the internet, online writers have turned to incitory headlines and content with the goal of receiving clicks. This is one source of fake news.

\par On April 27th, Facebook published an official transparency report on usage of the social media platform for \emph{information operations}. In this case, Facebook defines \emph{information operations} as ``Actions taken by governments or organized non-state actors to distort domestic or foreign political sentiment''. \cite{fb_info_ops} Information operations includes the dissemination of false news stories (via bots or otherwise) through social media to disinform the public. Here the motivation is political and the intent is to actively manipulate the beliefs of Facebook's users for personal gain. This amounts to lying, and Kantian ethics could not apply lying as a rule. \cite{kant} Therefore, the usage of fake news for information operations is entirely unethical.

\par Some publishers of fake news have a skewed self-perspective, viewing themselves as fiction writers or prophets. \cite{cbs_fn}


\subsubsection{Precedents}

\subsubsubsection{Burnett v. National Enquirer}

\par TODO: case against National Enquirer by Carol Burnett for libel. National Enquirer paid settlement out of court. Established the possibility of celebrities to sue tabloids for defamation but also established that the national enquirer is not a real newspaper and is not subject to restrictions on journalism.
\cite{burnett_v_ne}
