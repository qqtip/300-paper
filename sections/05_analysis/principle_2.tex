%!TEX root=../../main.tex

\subsection{Principle 2: The Client}

%============%
% TENET 2.07 %
%============%
\subsubsection{Tenet 2.07}

\tenetbox{Identify, \ul{document}, and \ul{report} \ul{significant issues} of \ul{social concern}, of which they are aware, in \ul{software} or related documents, to the employer or \ul{the client}.}

\par At the core of this tenet is the idea of a \ul{significant issue} of \ul{social concern} and how Facebook, as a software engineer, should address it.

%-------------------%
% SIGNIFICANT ISSUE %
%-------------------%
\subsubsubsection{Is False News a Significant Issue?}

\par According to the Oxford English Dictionary, an \ul{issue} is ``[a]n important topic or \emph{problem for debate}.'' \cite{oxford} That the issue is \ul{significant} emphasizes its importance.

\par In section \ref{disinformation}, we discussed the negative effects of false news: that its spread brings with it disinformation, confusion, and chaos. These effects are unwelcome and harmful. If false news is ``unwelcome and harmful,'' then according the the Oxford English Dictionary, false news is a \emph{problem}. \cite{oxford}

\par To \emph{debate} is ``to argue about,'' \cite{oxford} so for a problem to be \emph{debated} is for the problem to be argued about. In section \ref{perspectives}, we touched upon some of the many different perspectives on false news. While this list is not exhaustive, we have already established that false news is \emph{debated}.

\par Because false news is a \emph{debated problem}, it is a \ul{significant issue}.

%----------------%
% SOCIAL CONCERN %
%----------------%
\subsubsubsection{Social Concern with False News}

\par A \ul{concern} is ``[a] matter of interest or importance to \emph{someone},'' \cite{oxford} so to be of \ul{\emph{social} concern} is to be of importance to \emph{society}.

\par The Pew Research Center found that ``prominent internet analysts and the \emph{public at large have expressed increasing concerns} that the content, tone and intent of online interactions have undergone an evolution that threatens its future and theirs.'' \cite{pew_future} Here ``the public'' refers to ``the people of a country [or] state'' \cite{merriam_webster} --- in other words, \emph{society}. \cite{oxford}

\par If society has expressed concern over false news, then false news is of \ul{social concern}.

\par So in our domain, the \ul{significant issue of social concern} refers to the \ul{issue of false news}.

%---------------------%
% RELATED TO SOFTWARE %
%---------------------%
\subsubsubsection{False News and Software}

\par James Carson of The Telegraph states that ``[b]efore the Internet, publishing fake news and gaining an audience that could be monetised was nearly impossible.'' \cite{telegraph_fn}  And according to Hunt Allcott and Matthew Gentzk of Stanford and NYU, ``much of [false news'] circulation is on Facebook (and other social media) news feeds.'' \cite{socialmedia_fn}

\par If false news has little to no potential for growth outside of the Internet and Facebook's News Feed provides a primary avenue for false news to spread, false news depends on, or is at least linked to, Facebook's News Feed.

\par As determined in section \ref{se_system}) News Feed is a \emph{software system}. False news is related to News Feed and is therefore a problem related to software. The \ul{software} at hand is \ul{News Feed}.

%----------------------%
% ADDRESSING THE ISSUE %
%----------------------%
\subsubsubsection{Addressing False News}

\par We now have that Tenet 2.07 of the Code of Ethics would require Facebook, the software engineer, to ``[i]dentify, document, and \ul{report} [false news] to the employer or the client.'' \cite{se_code}

\par To \ul{report} is to ``\emph{[g]ive a... written account} of something that one has... \ul{investigated}.'' \cite{oxford} \emph{Giving a written account} falls into a similar vein as \ul{documentation}, but does involve \emph{sharing} the documents. In this domain, to ``share'' something is to ``[p]ost or repost (something) on a social media website or application.'' \cite{oxford} So to \ul{report} is to \ul{investigate and post written documentation}.

%-------------------%
% WHO IS THE CLIENT %
%-------------------%
\subsubsubsection{Facebook's Client}
\label{mission}

\par Taken from its official website, Facebook's mission ``is to give people the power to share and make the world more open and connected. People use Facebook to stay connected with friends and family, to discover what's going on in the world, and to share and express what matters to them.'' \cite{fb_newsroom}

\par According to its mission statement, Facebook aims to serve the people who use its software and not its customers. So even though Facebook's revenue comes from advertisers, \cite{fb_4q2016} Facebook's \ul{clients} are its \ul{users}.

%----------------------%
% DOMAIN SPECIFIC RULE %
%----------------------%
\dsr{2.07}{Facebook must identify and investigate the issue of false news in its News Feed and post written documentation to its users.}

\par We have established our issue of false news. We can now examine Facebook's actions and determine if Facebook performed a satisfactory identification and successful investigation. In addition, we will determine if Facebook posted documentation for its users. If so, Facebook has acted and is acting ethically by the Code.

%==========%
% IDENTIFY %
%==========%
\subsubsection{Facebook's Identification}

\par The first thing we must examine is whether Facebook successfully identified the issue of false news. For Facebook to be ethically successful, its actions must be professional and not \emph{negligent}, where ``negligence'' is ``failing to exercise the care expected of a reasonably prudent person in like circumstances.'' \cite{merriam_webster}

%------------------%
% SUCCESS CRITERIA %
%------------------%
\subsubsubsection{Satisfactory Identification}

\par Before proceeding, we should determine success criteria: specifically, what constitutes a \emph{satisfactory identification}.

\par To be ``satisfactory'' is to ``[f]ulfill expectations.'' \cite{oxford} Facebook is a software engineer and maintains its software News Feed. Tenet 3.07 of the Code of Ethics states that software engineers shall ``[s]trive to fully understand the specifications for software on which they work.'' \cite{se_code}

\par We can then expect Facebook to identify issues with its software before those outside of its organization.

\par At this time, Facebook received a large dose of media attention focused on false news. \cite{tc_snowden} There is no question whether Facebook is or is not \emph{currently} aware that false news was an issue on its News Feed. False news on Facebook received a large dose of media attention following Election Day, so even those outside of the organization are aware. \cite{tc_snowden}

\par We must then first determine whether Facebook recognized that false news was an issue on News Feed \emph{before} it received media attention. If not, Facebook did not take proper care and was therefore negligent.

%-----------%
% RECOGNIZE %
%-----------%
\subsubsubsection{When Did Facebook Recognize the Issue?}
\label{recognize}

\par On December 15, Facebook posted a News Feed status update to its Newsroom titled ``News Feed FYI: Addressing Hoaxes and Fake News.'' \cite{fb_hoaxes_2016} This update served to inform users about Facebook's new policies in response to false news and was the first update Facebook had posted about News Feed following the election. \cite{fb_hoaxes_2016}

\par This is not the first of such updates. Facebook has been publishing ``News Feed FYI'' updates since August 2013. \cite{fb_newsroom} Many record minor algorithmic improvements such as ``Showing More High Quality Content.'' \cite{fb_newsroom} 

\par But Facebook has also posted updates about ``Cleaning Up News Feed Spam.'' \cite{fb_newsroom} Upon further inspection, we find that Facebook published its first update addressing ``Showing Fewer Hoaxes'' in January 2015. \cite{fb_hoaxes_2015} Here Facebook defines ``hoaxes'' as ``a form of News Feed spam that includes scams... or deliberately false or misleading news stories.'' \cite{fb_hoaxes_2015} In other words, false news.

\par So while Facebook may have doubled its efforts in response to the media spotlight, \cite{guardian_steps} Facebook has been addressing false news since long before pressure was placed on them.

\par Facebook was not negligent enough to overlook such a serious issue with News Feed, but we can now determine if Facebook's response was ``reasonably prudent.''

%---------%
% GRAVITY %
%---------%
\subsubsubsection{Did Facebook Recognize the Gravity of its Situation?}

\par The Oxford English Dictionary likens ``prudence'' to ``cautiousness.'' \cite{oxford} So to determine if Facebook acted prudently, we will determine if Mark Zuckerberg recognized the gravity of the situation and acted with caution.

\par In his post, Mark Zuckerberg shared that ``Personally, I think the idea that fake news on Facebook — of which it's a small amount of content — influenced the election in any way is a pretty crazy idea.'' \cite{zuck_resp_a}

\par TechCrunch's Jordan Crook responded by accusing Zuckerberg of attempting to ``diminish fake news on Facebook as an issue'' \cite{tc_responsibility} and cited that ``more than half of U.S. adults get their news from social media'' \cite{tc_responsibility} and ``most students have trouble distinguishing fake news from real news.'' \cite{tc_responsibility} Crook was not the only one who expressed concerns. \cite{wp_russia}

\par But in January 2017, Stanford published a study that ``examine[d] fake news and the 2016 presidential election.'' \cite{stanford_election} This study proposed ``that false news stories may not have been as persuasive and influential as is often suggested.'' \cite{stanford_election}

\par Even then, Facebook proceeded with great caution: Despite ``growing pressure on Facebook over fake news... the company has appeared resistant.'' \cite{guardian_steps} Facebook enacted multiple measures to combat false news following the election, \cite{tc_downranks} but Zuckerberg continued to stress that ``we must be extremely cautious.'' \cite{zuck_resp_a}

\par So although the situation may have been less severe than anticipated, Facebook had still exercised extra caution, demonstrating professionalism.

%=============%
% INVESTIGATE %
%=============%
\subsubsection{Facebook's Investigation}

\par To ``investigate'' something is to ``[c]arry out a systematic or formal inquiry to discover and examine the facts of [that thing] so as to establish the truth.'' \cite{oxford} Because it published a report of its findings, we know that Facebook performed an investigation. \cite{fb_info_ops} To determine if the investigation was successful, we will ask if it discovered facts and established the truth.

%------%
% WHAT %
%------%
\subsubsubsection{What was being Investigated?}

\par In the wake of the election, Facebook received ``allegations its distribution of fake news helped elect Donald Trump'' \cite{tc_ai} --- allegations tied to a Russian propaganda effort. \cite{wp_russia} 

\par Although Zuckerberg had initially dismissed the validity of such a bold claim, \cite{zuck_resp_a} former FBI director James Comey confirmed in March that the FBI was investigating this case. \cite{comey} That Facebook's social network could have enabled these efforts could have conceivably placed liability in Facebook's hands.

%---------%
% RESULTS %
%---------%
\subsubsubsection{Results of the Investigation}
\label{investigation}

\par On April 27, Facebook released an extensive report on what it called ``information operations,'' which it defined as ``[a]ctions taken by governments or organized non-state actors to distort domestic or foreign political sentiment.'' \cite{fb_info_ops} Facebook stated that it ``is not in a position to make definitive attribution to the actors sponsoring this activity,'' \cite{fb_info_ops} but that ``[their] data does not contradict the attribution provided by the U.S. Director of National Intelligence.'' \cite{fb_info_ops}

\par This correlated with rumors circulating at the election's close. \cite{wp_russia,stanford_election} 

\par While Facebook did not necessarily ``discover the truth,'' it established \emph{certain} truths that had previously been based on rumor. \cite{cnn_fight} In addition, Facebook uncovered patterns in human-piloted amplification accounts (these are non-bot accounts that spread false news \cite{fb_info_ops}) that allowed the company to more efficiently combat false news.

\par By discovering patterns, Facebook performed an \emph{useful} investigation. The data it collected can now also be used to the advantage of its users (in a utilitarian sense). But because it could not determine a central actor, \cite{fb_info_ops} the investigation was not a complete success.

%========%
% REPORT %
%========%
\subsubsection{Facebook's Documentation}

\par We know now that Facebook performed a partially successful investigation following its satisfactory identification of the false news issue. Now we must determine if Facebook posted proper written documentation to its users.

\par Documentation, outside of software, is ``[m]aterial that provides official information or evidence or that serves as a record.'' \cite{oxford} For something to be written generally means for something to be ``set down in writing,'' \cite{merriam_webster} but in the field of computing it can mean ``to introduce (information) into the storage device or medium of a computer.'' \cite{merriam_webster} Because Facebook is a ``[s]oftware engineer... who contribute[s] by direct participation... to... software systems,'' \cite{se_code} and because ``software'' is tied to ``computer programs,'' \cite{merriam_webster} in our domain we will define ``written documentation'' as ``documentation accessible by a computer program.''

\subsubsubsection{Documentation Available to Users}

\par As discussed in \ref{investigation}, Facebook prepared an extensive report on its investigation into information operations. The report is available at \cite{fb_info_ops}. 

\par In addition, Facebook has been posting updates to its News Feed in its ``News Feed FYI'' series since August 2013, as mentioned in section \ref{recognize}. These posts are collectively available on the Internet at \url{https://newsroom.fb.com/news/category/news-feed-fyi/}. Because \url{https://newsroom.fb.com} is an address accessible via the Internet, it is a website. \cite{oxford} A website is ``[a] set of related web pages,'' \cite{oxford} and a web page is a ``[a] hypertext \emph{document}.'' \cite{oxford} Therefore, these posts are \emph{documents} and are valid documentation. And as web pages, they are available to any user with access to a computer browser.

%============%
% CONCLUSION %
%============%
\subsubsection{Conclusion}

\par We have determined that Facebook was not negligent in that the company early on identified the issue of false news in its News Feed. In addition, Facebook addressed it with caution in an act of prudence.

\par After identifying the issue of false news, Facebook performed a partially successful investigation and posted thorough documentation. These documents are all easily available to its users.

\par Because Facebook did not discover an actor in its investigation into information operations, \cite{fb_info_ops }the company did not achieve a complete success. But Facebook sufficiently identified that false news was a significant issue of social concern, documented it and reported to its client. So by Tenet 2.07 of the Software Engineering Code of Ethics, Facebook acted in the best interests of its client and has thus acted ethically.