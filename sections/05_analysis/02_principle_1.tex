%!TEX root = ../../main.tex

\subsection{Principle 1: Public}

%------%
% 1.02 %
%------%
\subsubsection{Tenet 1.02}

\tenetbox{\ul{Moderate} the \ul{interests} of the \ul{software engineer}, the \ul{employer}, the \ul{client} and the \ul{users} with the \ul{public good}. \cite{se_code}}

\par In the false news debate, Mark Zuckerberg and Facebook's other employees are the \ul{software engineers} involved, as they design, develop, maintain, and test the social media platform on which false news circulates. As Facebook develops its own product, Mark Zuckerberg and these same software engineers also comprise the \ul{employer}. 

\par Facebook's users, which include both readers and purveyors of false news, comprise the \ul{users} in this situation. As the end customers of the platform, these users are also Facebook's \ul{clients}.

\par While Facebook affects much of the public due to its size and reach, here the most important aspect of \ul{public good} being affected is the wide disinformation of its \ul{users}.

\par To \emph{moderate} something means to preside over and to bring balance to its extremities. \cite{dictionary} The \ul{interests} of these disparate groups are their various wants and needs. For Facebook to moderate their interests means for Facebook to preside over and balance these various wants and needs. 

\dsr{Facebook should balance their needs with the needs and well-being of their users.}
% TODO

%------%
% 1.04 %
%------%
\subsubsection{Tenet 1.04}

\tenetbox{\ul{Disclose} to \ul{appropriate persons or authorities} any \ul{actual or potential danger} to \ul{the user, the public}, or the environment, that they reasonably believe to be \ul{associated with software or related documents}. \cite{se_code}}

In this situation, the \ul{appropriate persons or authorities} are the fact-checkers behind CrossCheck, \ul{actual or potential danger} refers to misinformation, and \ul{the user and public} are Facebook users and readers of news.

\dsr{blah}
% TODO
