%!TEX root = ../../main.tex

%------------%
% 1. PUBLIC  %
%------------%
\subsection{Principle 1: Public}

%------%
% 1.02 %
%------%
\subsubsection{Tenet 1.02}

\emph{\ul{Moderate} the \ul{interests} of the \ul{software engineer}, the employer, the client and the \ul{users} with the \ul{public good}. \cite{se_code}}

Mark Zuckerberg and the employees of Facebook are the \ul{software engineers}, the users are registered Facebook users and posters of fake news, and the \ul{public good} is being affected by widespread misinformation.

For Facebook to \emph{moderate} the conflicting \ul{interests} of these disparate groups means for facebook to 

%------%
% 1.04 %
\subsubsection{Tenet 1.04}

\emph{Disclose to \ul{appropriate persons or authorities} any \ul{actual or potential danger} to \ul{the user, the public}, or the environment, that they reasonably believe to be associated with software or related documents. \cite{se_code}}

In this situation, the \ul{appropriate persons or authorities} are the fact-checkers behind CrossCheck, \ul{actual or potential danger} refers to misinformation, and \ul{the user and public} are Facebook users and readers of news.


%-------------------------%
% 2. CLIENT AND EMPLOYER  %
%-------------------------%
\subsection{Principle 2: Client and Employer}

%------%
% 2.07 %
%------%
\subsubsection{Tenet 2.07}

\emph{Identify, document, and report significant \ul{issues of social concern}, of which they are aware, in \ul{software} or related documents, to the employer or the client. \cite{se_code}}

In this situation, the \ul{issues of social concern} are the widespread misinformation of the public and \ul{software} refers to the social media network platform.


%-------------%
% 3. PRODUCT  %
%-------------%
\subsection{Principle 3: Product}

%------%
% 3.03 %
%------%
\subsubsection{Tenet 3.03}

\emph{Identify, define and address ethical, economic, cultural, legal and environmental issues related to work projects. \cite{se_code}}

As software engineers, those involved are expected to understand that their work has implications in their world at large.


%----------------%
% 7. COLLEAGUES  %
%----------------%
\subsection{Principle 7: Colleagues}

%------%
% 7.04 %
%------%
\subsubsection{Tenet 7.04}

\emph{\ul{Review} the \ul{work} of \ul{others} in an \ul{objective}, \ul{candid}, and \ul{properly-documented} way. \cite{se_code}}

The \emph{work} here refers to articles published by the \emph{others} --- online journalists, real or fake. For Facebook to \emph{review} these articles would be to determine the accuracy of said articles. To be \emph{objective} or \emph{candid} in review means to analyze without bias. To be \emph{properly-documented} in review means to be open and transparent in process. \cite{dictionary}

%
% DOMAIN SPECIFIC RULE
%
\noindent \textbf{Domain Specific Rule}

Applying Tenet 7.04 to our domain states that Facebook should \emph{determine the accuracy of online publishers' news articles without bias and be transparent in their methods}.


%------%
% 7.08 %
%------%
\subsubsection{Tenet 7.08}

\emph{In situations outside of their own \ul{areas of competence}, call upon the opinions of \ul{other professionals} who have \ul{competence in that area}. \cite{se_code}}

Facebook software engineers' \emph{areas of competence} are related to their work --- software engineering. As this issue revolves around news, it is outside of their area of competence.

Those with \emph{competence in the area} of news would be journalists, who have dedicated their career to collecting and editing news. \cite{dictionary} Journalists, then are the \emph{other professionals} who are competent in this area.

%
% DOMAIN SPECIFIC RULE
%
\noindent \textbf{Domain Specific Rule}

Applying Tenet 7.08 to our domain suggests that Facebook should, \emph{in situations related to news, call upon the opinions of journalists who specialize in that area}. 
