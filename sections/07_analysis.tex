%!TEX root = ../main.tex

\section{Analysis}

%!TEX root = ../main.tex

\subsection{Definitions}

\begin{enumerate}

Software Engineer

Product

Users

False News:

\end{enumerate}


%!TEX root = ../main.tex

\subsection{Domain Specific Rules}

%------%
% 1.02 %
%------%
\subsubsection{Tenet 1.02}

\emph{\underline{Moderate} the \underline{interests} of the \underline{software engineer}, the employer, the client and the \underline{users} with the \underline{public good}. \cite{se_code}}

Mark Zuckerberg and the employees of Facebook are the \underline{software engineers}, the users are registered Facebook users and posters of fake news, and the \underline{public good} is being affected by widespread misinformation.

For Facebook to \emph{moderate} the conflicting \underline{interests} of these disparate groups means for facebook to 

%------%
% 1.04 %
%------%
\subsubsection{Tenet 1.04}

\emph{Disclose to \underline{appropriate persons or authorities} any \underline{actual or potential danger} to \underline{the user, the public}, or the environment, that they reasonably believe to be associated with software or related documents. \cite{se_code}}

In this situation, the \underline{appropriate persons or authorities} are the fact-checkers behind CrossCheck, \underline{actual or potential danger} refers to misinformation, and \underline{the user and public} are Facebook users and readers of news.

%------%
% 2.07 %
%------%
\subsubsection{Tenet 2.07}

\emph{Identify, document, and report significant \underline{issues of social concern}, of which they are aware, in \underline{software} or related documents, to the employer or the client. \cite{se_code}}

In this situation, the \underline{issues of social concern} are the widespread misinformation of the public and \underline{software} refers to the social media network platform.

%------%
% 3.03 %
%------%
\subsubsection{Tenet 3.03}

\emph{Identify, define and address ethical, economic, cultural, legal and environmental issues related to work projects. \cite{se_code}}

%------%
% 7.04 %
%------%
\subsubsection{Tenet 7.04}

\emph{\underline{Review} the \underline{work} of \underline{others} in an \underline{objective}, \underline{candid}, and \underline{properly-documented} way. \cite{se_code}}

The \emph{work} here refers to articles published by the \emph{others} --- online journalists, real or fake. For Facebook to \emph{review} these articles would be to determine the accuracy of said articles. To be \emph{objective} or \emph{candid} in review means to analyze without bias. To be \emph{properly-documented} in review means to be open and transparent in process. \cite{dictionary}

In applying Tenet 7.04 to our domain, it states that Facebook should \emph{determine the accuracy of online publishers' news articles without bias and be transparent in their methods}.

%------%
% 7.08 %
%------%
\subsubsection{Tenet 7.08}

\emph{In situations outside of their own \underline{areas of competence}, call upon the opinions of \underline{other professionals} who have \underline{competence in that area}. \cite{se_code}}

Facebook software engineers' \emph{areas of competence} are related to their work --- software engineering. As this issue revolves around news, it is outside of their area of competence.

Those with \emph{competence in the area} of news would be journalists, who have dedicated their career to collecting and editing news. \cite{dictionary} Journalists, then are the \emph{other professionals} who are competent in this area.

In applying Tenet 7.08 to our domain, it suggests that Facebook should, \emph{in situations related to news, call upon the opnions of journalists who specialize in that area}. 


\subsection{The Intent of False News Publishers}

\par Once we have distinguished false news from satire and opinion pieces, it is important to discuss the intent of those who publish such stories. Unfortunately, there is no catch-all case for these publishers.

\par Barring AdBlock, it is not possible to browse the internet without being bombarded by (possibly-targeted) ads. As internet usership grew, so did the efficacy of ad space on the internet. Ads are now widespread and the ultimate goal for some websites is monetization through adspace. \cite{bbc_clickbait} So for many publishers of false news, the motivation is financial --- the intent is to receive clicks. Due to the capitalization of the internet, online writers have turned to incitory headlines and content with the goal of receiving clicks. This is one source of fake news.

\par On April 27th, Facebook published an official transparency report on usage of the social media platform for \emph{information operations}. In this case, Facebook defines \emph{information operations} as ``Actions taken by governments or organized non-state
actors to distort domestic or foreign political sentiment''. \cite{fb_info_ops} Information operations includes the dissemination of false news stories (via bots or otherwise) through social media to disinform the public. Here the motivation is political and the intent is to actively manipulate the beliefs of Facebook's users for personal gain. This amounts to lying, and Kantian ethics could not apply lying as a rule. \cite{kant} Therefore, the usage of fake news for information operations is entirely unethical.

\par Some publishers of fake news have a skewed self-perspective, viewing themselves as fiction writers or prophets. \cite{cbs_fake_news}

%
% PRECEDENTS
%
\subsection{Precedents}

\subsubsection{Burnett v. National Enquirer}

\par TODO: case against National Enquirer by Carol Burnett for libel. National Enquirer paid settlement out of court. Established the possibility of celebrities to sue tabloids for defamation but also established that the national enquirer is not a real newspaper and is not subject to restrictions on journalism.
\cite{burnett_v_ne}
