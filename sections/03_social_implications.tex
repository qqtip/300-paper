%!TEX root = ../main.tex

\section{Social Implications}

%----------------%
% DISINFORMATION %
%----------------%

% optional [title] for table of contents due to appearance of protected footnote in subsection title 
\subsection[Disinformation]{Disinformation\footnote{Here we will refer to Merriam-Webster to define \emph{disinformation}: false information deliberately and often covertly spread... in order to influence public opinion or obscure the truth. \cite{dictionary} This definition serves to distinguish \emph{disinformation} from \emph{misinformation}.}}

\par The rise of false news has caused enough of a stir that multiple online journals have documented its rise. \cite{telegraph_fake_news,npr_fake_news}

\par Allowing false news to proliferate on social media platforms such as Facebook can be seen as enabling disinformation of the public. \cite{tc_facebook_responsibility} The PEW Research Center reports that, as of May 2016, 44\% of American adults get news from Facebook. \cite{pew_social_media_news} In addition, 64\% of Americans believe that false news is causing confusion about current events. \cite{pew_fn_confusion}

\par Some journalists have claimed that the spread of false news on Facebook influenced the 2016 United States presidential election results. \cite{npr_fake_news}

\par A rigorous Stanford study has suggested that Facebook's involvement in the election results is heavily overstated and should not be held directly responsible for their sharing. \cite{stanford_fake_news_study}

\par However, some\footnote{Edward Snowden.} suggest that although the power of false news may be overestimated, the power of social media networks on the public should not be. \cite{tc_snowden_fb}

%------------%
% CENSORSHIP %
%------------%
\subsection{Censorship}

\par Between false news, satire, sensationalism, and opinionated journalism, there are many different non-overlapping interpretations of ``fake news''. \cite{npr_fake_news} This makes it difficult to objectively qualify what should and should not be regulated.

\par Regulation of media and corporate censorship --- as opposed to governmental censorship --- has been addressed since 1969. \cite{silent_screen} Facebook, as a corporation, is toeing the line in taking a proactive stance in the matter.

\par The regulation or systematic removal of anything published on the internet has also been interpreted as censorship and a curb on free speech. \cite{eff_cali_bill_distrastrous}

\vfill