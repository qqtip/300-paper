%!TEX root = ../main.tex

\section{Social Implications}


The rise of fake news has caused enough of a stir that multiple online journals have written articles on the subject. \cite{tc_what_is_fake_news,telegraph_fake_news,npr_fake_news} Even Facebook itself published tips on how to its help center \cite{fb_spot_fake_news} as well as a feature for users to report any article as "fake news". \cite{tc_fb_down_ranks_fake_news}

Allowing fake news to proliferate on social media platforms such as Facebook can be seen as enabling misinformation of the public, \cite{tc_facebook_responsibility} and according the PEW Research Center, 44\% of American adults get news from Facebook. \cite{pew_social_media_news} 

64\% of Americans believe that fake news is causing confusion around current events. \cite{pew_fn_confusion}

Publishers of fake news articles do not always have malicious intent; \cite{cbs_fake_news} many are financially motivated. \cite{fb_addressing_hoaxes} Any action affecting the proliferation of fake news would affect the economic stability of its creators.

% Facebook's involvement in the election results is heavily overstated and should not be held directly responsible for their sharing. \cite{stanford_fake_news_study}
%Former CIA employee Edward Snowden suggests that, although the power of fake news may be overestimated, the power of social media networks on the public should not be. \cite{tc_snowden_fb}

The systematic removal of anything published on the internet has also been interpreted as censorship and a curb on free speech. \cite{eff_cali_bill_distrastrous} The implementation of these platforms may diminish freedom on the internet.

