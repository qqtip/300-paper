%!TEX root = ../main.tex

\section{Social Implications}
\label{social}

\par Because the false news epidemic is a new phenomenon (at least on this scale), Facebook is venturing into the unknown in addressing it. Facebook is a leader in the tech world -- its social networking service is the third most popular website on the internet. \cite{alexa} How Facebook handles this situation may very well set a precedent for how others handle it in the future.

%================%
% DISINFORMATION %
%================%

% optional [title] for table of contents due to appearance of protected footnote in subsection title 
\subsection[Disinformation]{Disinformation\footnote{Here we will refer to Merriam-Webster to define \emph{disinformation}: false information deliberately and often covertly spread... in order to influence public opinion or obscure the truth. \cite{merriam_webster} This definition serves to distinguish \emph{disinformation} from \emph{misinformation} by its deliberate intent.}
} 
\label{disinformation}

\par The proliferation of false news on social media platforms directly affects the public at large. 62\% of American adults get news from social media, so it is easy for ``alternative facts'' to deceive large populations.  In fact, 64\% of American adults believe that false news is causing confusion about current events. \cite{pew_confusion}

\par Some journalists have claimed that the spread of false news on Facebook influenced the 2016 United States presidential election results. \cite{npr_fn} In November 2016, BuzzFeed News\footnote{The author does not condone reading BuzzFeed News.} posted an analysis showing that false news articles published on Facebook during the election outperformed news articles published by mainstream news outlets.\footnote{The study revealed specifically that the top 5 false news articles published on Facebook during the final three months of the election received more engagement than the top 5 news articles published by mainstream news outlets. Engagement was quantitatively measured in shares, reactions, and comments. \cite{buzzfeed_analysis}} \cite{buzzfeed_analysis}

\par However, a rigorous Stanford study has suggested that Facebook's involvement in the election results is heavily overstated and should not be held directly responsible for their sharing. \cite{stanford_election}

\par Edward Snowden still suggests that although the power of false news may be overestimated, the power of social media on the public should not be. \cite{tc_snowden}
\par Finally, through word-of-mouth and traditional media, false news can even disperse through populations beyond the Internet and cause very real world consequences. The Pizzagate scandal is a prime example.\footnote{Pizzagate was a scandal revolving around a fabricated conspiracy theory that made international waves during the 2016 election. The theory alleged that a human trafficking and child-sex ring was being conducted by the U.S. Democratic Party out of the basement of a Washington D.C. pizza restaurant. Unlikely as it sounds, one firm, armed believer visited the restaurant to investigate and fired a shot before realizing that the restaurant did not have a basement.\cite{ny_pizzagate}} \cite{cbs_fn}

\vfill

%============%
% CENSORSHIP %
%============%
\subsection{Censorship}

\par Between false news, satire, sensationalism, and opinionated journalism, there are many different non-overlapping interpretations of ``fake news''. \cite{npr_fn} This makes it difficult to objectively qualify what should and should not be regulated.

\par Regulation of media and corporate censorship --- as opposed to governmental censorship --- has been addressed since 1969. \cite{silent_screen} Facebook, as a corporation, is toeing the line in taking a proactive stance in the matter.

\par The regulation or systematic removal of anything published on the internet has also been interpreted as censorship and a curb on free speech. \cite{eff_bill}

%\vfill