%!TEX root = ../main.tex

\section{Social Implications}


% social values at stake
% economic for purveyors of fake news
% freedom of speech



According the PEW Research Center, 44\% of American adults get news from Facebook. \cite{ew_social_media_news} 

Publishers of fake news articles do not always have malicious intent. \cite{cbs_fake_news}

This has even led to conjecture that the platform's enabling of misinformation may have influenced the results of the presidential election. \cite{bbc_facebook_fake_news_crisis, wp_russian_election_propaganda} 

Facebook's involvement in the election results is heavily overstated and should not be held directly responsible for their sharing. \cite{stanford_fake_news_study}
Censorship
Facebook uses AI instead of human curators to remove bias, denies bias but fired curating team \cite{tc_facebook_ai}


Censorship is not the answer to fake news. \cite{eff_cali_bill_distrastrous}

Former CIA employee Edward Snowden suggests that, although the power of fake news may be overestimated, the power of social media networks on the public should not be. \cite{tc_snowden_fb}

Many online journals have posted articles on fake news; \cite{tc_what_is_fake_news,telegraph_fake_news,npr_fake_news} even Facebook itself published tips on how to its help center \cite{fb_spot_fake_news} as well as a feature for users to report any article as "fake news". \cite{tc_fb_down_ranks_fake_news}