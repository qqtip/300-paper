%!TEX root = ../main.tex

\section{Perspectives}
\label{perspectives}

%-------------------%
% ARGUMENTS AGAINST %
%-------------------%
\subsection{Opposing Arguments}

\subsubsection{Removing fake news is a curb on free speech}

\par Using the law to ban fake news is censorship and a violation of free speech. Furthermore, banning fake news will cause political chaos as politicians turn to name-calling and accusations. \cite{eff_bill}
\par In addition, compelling social media outlets to remove fake news is tantamount to censorship. Though it is censorship by a corporation and not the government, it is still a result of government action. \cite{wp_germany}

\subsubsection{Fact-checking cannot be inherently neutral}

\par People have very different opinions about what constitute{}s ``fake news,'' and ``fact-checking'' is not inherently neutral. Introducing a human element cannot be achieved without also introducing bias. \cite{vox_sentences} In fact, since Facebook is the entity deciding what is ``fake news,'' this is tantamount to liberal censorship. \cite{milo_censor}

\subsubsection{The selected fact-checkers are not credible}

\par ``When the media outlets performing the work of fact-checking do not themselves have credibility, no one is going to accept their conclusions. And that ship has already sailed.'' \cite{vox_sentences}

%---------------%
% ARGUMENTS FOR %
%---------------%
\subsection{Supporting Arguments}

\subsubsection{The goal is not to remove stories}

\par The goal of Facebook's initiative is to focus on removing clearly untrue stories and not opposing opinions --- Facebook will continue to rely on unbiased third parties. \cite{tc_downranks}

\subsubsection{The strategy does not involve direct censorship}

\par The strategy is centered around flagging untrustworthy content, not removing content. While accounts are being removed, these accounts are ones believed to be constructed with the sole purpose of spreading false news. \cite{tc_downranks}

\subsubsection{Facebook is being responsible}

\par Facebook is one of the largest forums for civic discourse (including of politics). By addressing the issue of false news, Facebook is owning its role in shaping public perception of current events. \cite{cnn_fight}

%
\vfill