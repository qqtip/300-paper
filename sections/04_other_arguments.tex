%!TEX root = ../main.tex

\section{Other's Arguments}

%---------------%
% ARGUMENTS FOR %
%---------------%
\subsection{Supportive}


\subsubsection{The goal is not to remove stories}

The goal of the initiative is to focus on removing clearly untrue stories and not opposing opinions -- Facebook will continue to search for unbiased third parties. \cite{inc_fb_transparency}

\subsubsection{The strategy does not involve direct censorship}

The strategy is about flagging untrustworthy content, not removing content. \cite{inc_fb_transparency}


%-------------------%
% ARGUMENTS AGAINST %
%-------------------%
\subsection{Opposing}


\subsubsection{Banning fake news is a curb on free speech}

Using the law to ban fake news is censorship and a violation of free speech --- banning fake news will cause political chaos as politicians turn to name-calling and accusations. \cite{eff_cali_bill_distrastrous}

\subsubsection{Compelling an organization to remove fake news is a curb on free speech}

Compelling social media outlets to remove fake news is a curb on free speech. \cite{wp_germany_fake_news_law}


\subsubsection{Fact-checking cannot be inherently neutral}

People have very different opinions about what constitutes ``fake news'', and ``fact-checking'' is not inherently neutral. Introducing a human element will introduce bias. \cite{vox_sentences_fn} \cite{inc_fb_transparency}

\subsubsection{Selecting uncredible fact-checkers is unacceptable}

``When the media outlets performing the work of fact-checking do not themselves have credibility, no one is going to accept their conclusions. And that ship has already sailed.'' \cite{vox_sentences_fn}
%Critics want more transparency \cite{inc_fb_transparency}}

