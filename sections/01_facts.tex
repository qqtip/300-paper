%!TEX root = ../main.tex


%-------%
% FACTS %
%-------%
\section{Facts of the Matter}


Since mid-2016, ``fake news'' has found itself at the forefront of current events. \cite{telegraph_fake_news} Donald Trump popularized the term during the American presidential election where he liberally to describe various news sources that disagreed with him. \cite{tc_what_is_fake_news} It has become something of a catch phrase of his --- he tweeted the term over 15 times in February, "(10 times in all caps)". \cite{npr_fake_news}

This usage has only served to obscure the original meaning of fake news more important to a journalistic perspective: news that intentionally deceives readers for personal or political gain \cite{tc_what_is_fake_news }\cite{npr_fake_news} --- in other words, propaganda. \cite{telegraph_fake_news}

While propaganda is not a new phenomenon, modern technology has pushed it to new levels of efficacy and profitability with the Internet having made it easier for organizations and individuals to publish fake news by removing distribution costs and allowing publishers to avoid regulation. \cite{telegraph_fake_news}. In addition, social media sharing has made it easier to gain a wide readership, further enhanced by the usage of bot accounts. \cite{cbs_fake_news}.

Fake news has become a major issue for social media giant Facebook, leading to conjecture that the platform's enabling of misinformation influenced the results of the presidential election \cite {bbc_facebook_fake_news_crisis} \cite{wp_russian_election_propaganda} and even allegations that Facebook enjoys the increased readership and profitability of fake news. \cite{tc_facebook_responsibility}

In reponse, Facebook has had to publish a set of tips on how to spot fake news to its help center. \cite{fb_spot_fake_news} In addition, Facebook has added a feature for users to report any article as "fake news". \cite{tc_facebook_down_ranks_fake_news}