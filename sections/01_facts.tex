%!TEX root = ../main.tex


%-------%
% FACTS %
%-------%
\section{Facts of the Matter}

%
% TODO:
%
% research tabloids and compare them to false news, rather than(/in addition to) propaganda
%

Donald Trump popularized the term ``fake news'' during the 2016 American presidential election, \cite{tc_what_is_fake_news} but his usage has only served to obscure the original meaning of ``fake news'': news that intentionally deceives readers for personal or political gain. \cite{tc_what_is_fake_news, npr_fake_news} To distinguish these meanings, Facebook has turned to using the term ``false news'' to describe these articles. \cite{fb_spot_fake_news}

While false news is not a new phenomenon --- disinformation has been historically used for propaganda \cite{bbc_fn_propaganda} --- modern technology has pushed false news to new levels of efficacy and profitability by removing regulations and easing distribution. \cite{telegraph_fake_news}. Tech giants Facebook and Google play a major part in the spread of false news due to their sharing functionality and ad space, which aide the propagation of false news stories. \cite{cbs_fake_news, tc_google_took_down_ads}

The size of Facebook has even sparked debate over whether its enabling of misinformation might have influenced the results of the presidential election \cite{bbc_facebook_fake_news_crisis, wp_russian_election_propaganda} and even allegations that Facebook enjoys the increased readership and profitability of false news. \cite{tc_facebook_responsibility}

In response, Google has taken down billions of ads from its network \cite{tc_google_took_down_ads} and Facebook has added support for flagging false news as well as publishing various guides to users regarding hoaxes and clickbait. \cite{tc_fb_down_ranks_fake_news,fb_reduce_clickbait,fb_addressing_hoaxes,fb_spot_fake_news} However, these efforts have encountered criticism by those comparing it to censorship. \cite{vox_sentences_fn}

In a further push to combat the spread of false news, Google and Facebook have partnered with First Draft News to create CrossCheck. \cite{crosscheck_launch} Crosscheck is a platform supported by various news sources --- such as Agence France-Presse and Buzzfeed News --- that aims to identify and expose misleading media such as news sites, photographs, and videos. \cite{tc_google_fb_partner,pcw_france_google_facebook}