%!TEX root = ../main.tex


%-------%
% FACTS %
%-------%
\section{Facts of the Matter}

Though Donald Trump popularized the term "fake news" during the 2016 American presidential election, \cite{tc_what_is_fake_news} his usage has only served to obscure the original meaning of fake news: news that intentionally deceives readers for personal or political gain. \cite{tc_what_is_fake_news, npr_fake_news}

While this is not a new phenomenon --- propaganda \cite{bbc_fn_propaganda} ---Modern technology has pushed fake news to new levels of efficacy and profitability by removing regulations and easing distribution. \cite{telegraph_fake_news}. Tech giants Facebook and Google play a major part in the spread of fake news due to their sharing functionality and ad space,  \cite{cbs_fake_news, tc_google_took_down_ads}

The size of Facebook has even sparked debate over whether its enabling of misinformation might have influenced the results of the presidential election \cite{bbc_facebook_fake_news_crisis, wp_russian_election_propaganda} and even allegations that Facebook enjoys the increased readership and profitability of fake news. \cite{tc_facebook_responsibility}

In response, Google has taken down billions of ads from its network \cite{tc_google_took_down_ads} and Facebook has added support for flagging fake news as well as publishing various guides to users regarding hoaxes and clickbait. \cite{tc_fb_down_ranks_fake_news,fb_reduce_clickbait,fb_addressing_hoaxes,fb_spot_fake_news} However, these efforts have encountered criticism from  

In a further push to combat the spread of fake news, Google and Facebook have partnered with First Draft News to create CrossCheck, \cite{crosscheck_launch} a platform whose goal is to identify and expose misleading media such as news sites, photographs, and videos by working with various news sources such as Agence France-Presse and Buzzfeed News. \cite{tc_google_fb_partner,pcw_france_google_facebook}