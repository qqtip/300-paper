%!TEX root = ../main.tex


%-------%
% FACTS %
%-------%
\section{Facts of the Matter}


Since mid-2016, ``fake news'' has found itself at the forefront of current events. \cite{telegraph_fake_news} Donald Trump popularized the term during the 2016 American presidential election where he liberally used it to describe various news sources that disagreed with him. \cite{tc_what_is_fake_news}

This usage has only served to obscure the original meaning of fake news: news that intentionally deceives readers for personal or political gain \cite{tc_what_is_fake_news }\cite{npr_fake_news} --- in other words, propaganda. \cite{telegraph_fake_news}

While propaganda is not a new phenomenon, modern technology has pushed it to new levels of efficacy and profitability: the Internet has made it easier for organizations and individuals to publish fake news by removing regulations and distribution. \cite{telegraph_fake_news}. In addition, social media's sharing functionality has made it easier to spread fake news and gain a wide readership. \cite{cbs_fake_news} Some sources of fake news further this by using social media bot accounts to share their articles. \cite{cbs_fake_news}

Fake news has become a major issue for tech giants Facebook and Google. \cite{bbc_facebook_fake_news_crisis}

Facebook added support for down ranking fake news --- initially had a human team curating but there was controversy, blah blah, had to fire bitches. \cite{tc_fb_down_ranks_fake_news}

Google took down 1.7 bill ads from its ad net2work in 2016 \cite{tc_google_took_down_ads}

and even allegations that Facebook enjoys the increased readership and profitability of fake news. \cite{tc_facebook_responsibility}

Google/FB partner for crosscheck \cite{tc_google_fb_partner}

Among the organizations involved with CrossCheck even included Buzzfeed News (at one point?) \cite{pcw_france_google_facebook}

