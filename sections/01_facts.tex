%!TEX root = ../main.tex

%=======%
% FACTS %
%=======%
\section{Facts of the Matter}

\subsection{The Rise of Fake News}
\label{facts_rise}

\par Donald Trump popularized the term \emph{fake news} during the 2016 United States presidential election, liberally using the term to describe major media outlets such as the New York Times and CNN. \cite{telegraph_fn} Unfortunately, his usage has obscured a more semantic meaning of \emph{fake news} relevant to our interests: stories that intentionally deceive readers for personal or political gain. \cite{npr_fn}

\par To distinguish these meanings, some (like Facebook) have turned to using the term \emph{false news} to describe these articles. \cite{guardian_tips} We will follow suit.

\par While false news is not a new phenomenon --- we used to call it propaganda --- modern technology has turned false news into a profitable venture. \cite{bbc_propaganda}

\subsection{Viral Infection}

\par Online social networks like Facebook and Twitter are standout examples of such technologies. Not only do social networks provide an inexpensive, unregulated platform for distribution, \cite{telegraph_fn} the users themselves help circulate articles by ``sharing'' them with connected users. \cite{telegraph_fn} 

\par Facebook, with 1.28 billion daily active users,\footnote{As of March 2017.} \cite{fb_newsroom} provides a vast network for distribution. Considering 2/3rds of its users get news from its social networking service, \cite{pew_news} Facebook plays a major role in the spread of false news.

\par The size of Facebook has even sparked debate over whether its enabling of disinformation influenced the results of the election. \cite{wp_russia,guardian_steps,tc_snowden} Mark Zuckerberg, CEO of Facebook, has even received allegations that Facebook enjoys the increased readership and profitability of false news. \cite{tc_responsibility}

\subsection{Facebook's Reply}

\par In response to this criticism, Facebook has enacted multiple measures to combat the spread of false news. \cite{fb_newsroom,fb_hoaxes_2016,guardian_tips,tc_downranks,tc_ai,telegraph_delete}

\par Initial efforts in 2016 included introducing algorithms to detect and reduce clickbait\footnote{Clickbait here is defined as links with titles that are misleading or withhold information to entice users to ``click'' the link. \cite{fb_hoaxes_2016}} and false news on Facebook's News Feed \cite{fb_hoaxes_2016} and adding a feature for users to flag false news articles.\footnote{On April 6, 2017, Facebook further published a set of tips for identifying false news. Hundreds of millions of users in 14 countries received notifications of their existence and were prompted to read them. \cite{guardian_tips}} \cite{tc_downranks} 

\par In early 2017, Facebook expanded its efforts by working alongside various news agencies and fact-checking organizations internationally to flag false news stories. \cite{guardian_steps} As of May 2017, Facebook has deleted tens of thousands of accounts that spread false news. \cite{telegraph_delete}

\vfill