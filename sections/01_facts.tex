%!TEX root = ../main.tex


%-------%
% FACTS %
%-------%
\section{Facts of the Matter}

\par Donald Trump popularized the term ``fake news'' during the 2016 United States presidential election, \cite{tc_what_is_fake_news} but his usage has obscured the original meaning of ``fake news'': articles that intentionally deceive readers for personal or political gain. \cite{tc_what_is_fake_news, npr_fake_news}

\par To distinguish these meanings, Facebook has turned to using the term ``false news'' to describe these articles. \cite{fb_spot_fake_news}

\par While false news is not a new phenomenon --- disinformation has been historically used as propaganda \cite{bbc_fn_propaganda} --- modern technology has pushed false news to new levels of efficacy and profitability by removing regulations and easing distribution. \cite{telegraph_fake_news}.

\par Social media outlets are an example of one such technology and are a powerful resource for purveyors of false news. They provide a low-cost means of distributing articles and, due to ``sharing'' functionality, users and bots can push these articles viral. \cite{cbs_fake_news}

\par Tech giant Facebook, who claims to have over a billion daily users on their social media platform, plays a major role in the spread of false news. \cite{cbs_fake_news} In fact, in November 2016, BuzzFeed News \footnote{The author does not condone reading BuzzFeed News.} posted an analysis showing that false news articles published on Facebook during the election outperformed news articles published by mainstream news outlets. \footnote{The study specifically showed that, in the last three months of the campaign, the top 5 false news articles published on Facebook received more engagement than the top 5 news articles published by recognized ``mainstream'' news outlets. Here engagement is quantitatively measured in ``shares'', ``reactions'', and ``comments''. \cite{buzzfeed_news_analysis}} \cite{buzzfeed_news_analysis}

\par These false news articles featured headlines such as
\emph{``WikiLeaks CONFIRMS Hillary Sold Weapons to ISIS... Then Drops Another BOMBSHELL! Breaking News''} and
\emph{``Pope Francis Shocks World, Endorses Donald Trump for President, Releases Statement''}. \cite{buzzfeed_news_analysis}

\par The size of Facebook has even sparked debate over whether its enabling of disinformation might have influenced the results of the election. \cite{bbc_facebook_fake_news_crisis, wp_russian_election_propaganda} and even allegations that Facebook enjoys the increased readership and profitability of false news. \cite{tc_facebook_responsibility}

\par In response to this criticism, Facebook has enacted multiple measures to combat the spread of false news.

\par Early measures in late 2016 included reducing clickbait\footnote{Clickbait here is defined as links with titles that are misleading or withhold information to entice users to follow the link. \cite{fb_reduce_clickbait}} in users' News Feeds \cite{fb_reduce_clickbait} and adding support for users to flag false news articles.\footnote{On April 6, 2017, Facebook further published a set of tips for identifying false news. Hundreds of millions of users in 14 countries received notifications of its existence and were prompted to read them. \cite{fb_spot_fake_news,guardian_fb_tips}} \cite{fb_addressing_hoaxes,tc_fb_down_ranks_fake_news} 

\par More recent measures include working alongside various news agencies and fact-checking organizations to flag false news stories \cite{guardian_steps} and even deleting tens of thousands of accounts that spread false news. \cite{telegraph_fb_delete}