%!TEX root=../main.tex

%=======%
% TITLE %
%=======%

\title{\vfill Facebook's Fake News Feed\\
\vspace{8pt}
\normalsize{}
}

\author{Minh-Quan Tran\\
\normalsize{CPE 300: Professional Responsibilities}\\
\normalsize{California Polytechnic State}\\
\normalsize{Dr. Clark Savage Turner}
}

\maketitle


%==========%
% ABSTRACT %
%==========%
\vfill

\begin{abstract}
The spread of false news on Facebook has been a source of international controversy since the 2016 U.S. presidential election. \cite{telegraph_fn} In response to allegations that false news may have influenced the election's outcome, \cite{tc_snowden,stanford_election} Facebook has enacted various measures to suppress its appearance on its News Feed. \cite{fb_hoaxes_2016,guardian_tips} But is it ethical for Facebook to curate the news? Some journalists commend Facebook for taking ownership of its role in shaping current events. \cite{cnn_fight} Others denounce Facebook's efforts, citing bias \cite{vox_sentences} and calling it censorship. \cite{milo_censor} By identifying and addressing the danger of disinformation, Facebook is accepting responsibility for the safety and knowledge of its users. Therefore, Facebook's actions are in their users' best interests and are consistent with Principle 2 of the Software Engineering Code of Ethics. \cite{se_code} 
\end{abstract}

\thispagestyle{empty} 
\pagebreak