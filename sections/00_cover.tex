%!TEX root = ../main.tex

%=======%
% TITLE %
%=======%

\title{\vfill Facebook vs. Fake News\\
\vspace{8pt}
\normalsize{}
}

\author{Minh-Quan Tran\\
\normalsize{CPE 300: Professional Responsibilities}\\
\normalsize{California Polytechnic State}\\
\normalsize{Dr. Clark Savage Turner}
}

\maketitle


%==========%
% ABSTRACT %
%==========%
\vfill

\begin{abstract}
The spread of false news on Facebook has been a source of international controversy since the 2016 U.S. Presidential Election. \cite{telegraph_fake_news} In response to allegations that it may have influenced the election's outcome, \cite{tc_snowden_fb,stanford_fake_news_study} Facebook has enacted various measures to suppress false news. \cite{fb_addressing_hoaxes,guardian_fb_tips} But is it ethical for Facebook to effectively curate news? Some journalists commend Facebook for taking ownership of its role in shaping current events. \cite{cnn_fight} Others denounce their efforts,  citing bias \cite{vox_sentences_fn} and drawing comparisons to censorship. \cite{milo_censor} Tenet Facebook is addressing its responsibility for the safety and knowledge of the public. \cite{se_code} Facebook is then acting ethically.
\end{abstract}

\thispagestyle{empty} 
\pagebreak