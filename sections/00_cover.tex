%!TEX root = ../main.tex


%-------%
% TITLE %
%-------%

\title{\vfill Facebook vs. Fake News\\
\vspace{8pt}
\normalsize{}
}

\author{Minh-Quan Tran\\
\normalsize{CPE 300: Professional Responsibilities}\\
\normalsize{California Polytechnic State}\\
\normalsize{Dr. Clark Turner}
}

\maketitle


%----------%
% ABSTRACT %
%----------%
\vfill

\begin{abstract}
The spread of false news stories via Facebook has been a source of controversy since the 2016 U.S. Presidential Election. \cite{telegraph_fake_news} In response to allegations that its platform enables disinformation that may have influenced the election's outcome, \cite{tc_snowden_fb,stanford_fake_news_study} Facebook has enacted various measures to suppress false news. \cite{fb_reduce_clickbait,fb_addressing_hoaxes,fb_spot_fake_news} Is it ethical for Facebook to effectively curate news by curbing the growth of false news? Some journalists commend the tech giant for taking ownership of its role in shaping current events. \cite{cnn_fight} Others have denounced Facebooks efforts, citing the system's bias. \cite{vox_sentences_fn} Some even call it liberal censorship. \cite{milo_censor} According to the first principle of the ACM Software Engineering Code of Ethics, those at Facebook --- as software engineers --- are responsible for the safety and knowledge of the public. \cite{se_code} So in mitigating the disinformation of its users, Facebook is adhering to the Code and their actions are justified.
\end{abstract}

\thispagestyle{empty} 
\pagebreak