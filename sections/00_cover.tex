%!TEX root = ../main.tex


%-------%
% TITLE %
%-------%

\title{\vfill Facebook vs. Fake News\\
\vspace{8pt}
\normalsize{}
}

\author{Minh-Quan Tran\\
\normalsize{CPE 300: Professional Responsibilities}\\
\normalsize{California Polytechnic State}\\
\normalsize{Dr. Clark Turner}
}

\date{April 14, 2017}

\maketitle


%----------%
% ABSTRACT %
%----------%
\vfill

\begin{abstract}
The spread of false news stories via social network Facebook has been a source of controversy since the 2016 U.S. Presidential Election. \cite{telegraph_fake_news} In response to allegations that its platform enables misinformation that may have influenced the outcome of the election, \cite{tc_snowden_fb} Facebook has enacted multiple measures to mitigate the impact of disinformation on its users. \cite{fb_reduce_clickbait,fb_addressing_hoaxes,fb_spot_fake_news} In a recent push, Facebook (alongside Google) has partnered with First Draft News to create CrossCheck, a platform of self-selected news curators who would identify and expose various news sources. \cite{crosscheck_launch} Is it ethical for Facebook to effectively curate news by curbing the growth of false news stories? Software engineers have a responsibility to the safety and knowledge of the public, \cite{se_code} so it is ethical for Facebook to influence the visibility of news.
\end{abstract}

\thispagestyle{empty} 
\pagebreak