%!TEX root = ../main.tex

\subsection{Definitions}

\begin{enumerate}

\item{\textbf{Product}: The \emph{product} in this case will refer to social media platform Facebook, which is a piece of software developed by Mark Zuckerberg and his employees.
}

\item{\textbf{Software Engineer}: The \emph{software engineers} in this case are those who develop, test, and maintain Facebook: Mark Zuckerberg and his employees.
}

\item{\textbf{Users}: The \emph{users} in this case are people who have registered for Facebook and who browse the platform at least weekly.
}

\item{\textbf{Fake News:} One of the difficulties in identifying "fake news" is semantic: President Donald Trump's liberal usage the term during the 2016 Presidential Election has obscured its meaning. \cite{npr_fake_news} To remedy this, we will define the following:
\begin{enumerate}
    \item{\emph{False News}: A story that is not based in fact that may reports events that did not happen. A false news story is published with the intent to deceive. \cite{ted_becka}}
    \item{\emph{Satire}: A published story that satirizes news media. \cite{ted_becka}}
    \item{\emph{News with Opinion}: An editorial piece that covers news and introduces the bias of the writer. \cite{ted_becka}}
\end{enumerate}
}

\end{enumerate}